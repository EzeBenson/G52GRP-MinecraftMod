\chapter{Design and Implementation}

\section{Overview}
A discussion of the design of the system, and key factors that 
influenced our design. We should talk about the requirement to 
develop using Java, and the decision to use the MinecraftForge API.

\section{Using Java}
Minecraft is a game written in the Java language, and so it is 
only logical that when we mod the game it is required that we use 
Java. This links in with the API we use, which is also dependent 
on Java and specifically is made with support for Eclipse (in 
terms for example of online tutorials for setting up and coding 
in the Eclipse JDK environment). 

\section{Decision to use MinecraftForge API}
Before creating our mod we had to choose the method for modding 
the game. Our final choice was to use the MinecraftForge API, which 
is a mod/application layer for Minecraft which lets us create and 
run Minecraft mods. The decision to use MinecraftForge API was 
because of its versatility and usefulness. Furthermore, we decided 
to use MinecraftForge over other alternatives because, while other 
tools like ModLoader may have been arguably easier to use, MinecraftForge 
is much more suited for our purposes – namely, creating a more complex 
mod. This is because MinecraftForge has a lot more possibilities, whereas 
with ModLoader there is more limited scope for creating complex Minecraft 
mods. Aside from ModLoader and MinecraftForge, there wasn’t much else 
choice; these two mod tools are the most popular by far in the Minecraft 
community.  

\section{Discussion of the design of the system}
Because of the very nature of MinecraftForge and our design, 
all modifications are done by creating new classes – we are 
creating a mod that adds and extends from Minecraft, we are 
not creating a core mod. Because of this we have our mod in 
a new package: package org.educraft, which can be implemented 
by MinecraftForge and run in Minecraft. 

Our system design has proxy classes, which at the moment aren’t used 
for much but will be used later (when our mod is more developed) for 
purposes such as rendering entities in the Minecraft world (e.g. monsters, items). 
In the current dummy mod we have produced, we have a base class (DummyMod.java), 
which is the first class that MinecraftForge loads. The base class uses other 
classes such as DummyCoin, DummyAttackHandler etc. These separate classes 
control specific individual elements of the mod. For example, DummyCoin extends 
the item class, and is a custom item that we have created for use in our mod. 
Likewise, we use similar tactics in other classes, such as a ‘maths wand’ 
which extends the Minecraft Sword item, and has unique interactions with other 
aspects of the mod such as our dummy zombie. 

In its current state the mod is obviously fairly basic, but we have the 
foundations for an organised system design which can and will be the basis 
for a sophisticated and elegant mod. A key reason for this system design 
is that it follows the standard format of a MinecraftForge mod, meaning it 
is easy to understand, and is well organised. 

Examples of/ tutorials for MinecraftForge mods informed our decision to 
create our mod following this system design; examples can be found on the 
wiki page: http://www.minecraftforge.net/wiki/Tutorials
