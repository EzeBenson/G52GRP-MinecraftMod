\chapter{Initial prototypes}

\section{Introduction}
Prototyping is an important part of any software engineering project,
as it enables the development team to demonstrate their progress to
their client in a way that they can more easily understand, and also
enables the client to clarify what they want from the system
\cite{brooks1987,sommerville2011}.

In this chapter we describe the prototypes which have been used to
demonstrate EduCraft in its early development, along with how they were
received by those we showed them to. We first describe the mod which
was used in these demonstrations, and then explain the demonstrations
themselves.

\section{The `Dummy Mod'}
Both prototypes involved using something we called the `Dummy Mod'.
This was a simple mod which contained some basic custom items
which we believed would serve to demonstrate the goals which we were
aiming towards. Neither of the prototypes actually included any
elements that would be used in the finished product; rather, they served
simply to \textit{illustrate} what we hoped to achieve.

The premise of Dummy Mod was simple: we provided the player with a
customised weapon item, and the means to generate a
number of modified enemies. When an enemy was slain with this custom weapon,
instead of dropping its usual items, instead it dropped a special
coin. We also gave the means to combine nine of these coins into a
`pile of coins', and to split a pile of coins into nine indvidual coins.

These basic implementations demonstrate what we hope to achieve in the
finished product, but with much less work: the code used to create the
items was minimal, pre-existing textures from within the game were
used instead of custom-made ones, and no special mathematical concepts
were implemented. The purpose was simply to illustrate that we could,
in principle, make a Minecraft mod which allowed players to use special
tools to collect particular items from enemies, which could then be
combined together.

\section{Demonstration to supervisor}
The first demonstration was carried out in a formal meeting with our
supervisor. This demonstration was very basic: we did not intent to present
anything approaching a working `level' of the game, but simply wished
to demonstrate that we knew how to work with Minecraft to produce mods.

\section{Demonstration at Southwold Primary School}
The second demonstration was much more involved, and was given to the
maths co-ordinator of Southwold Primary School, who would be responsible
for deciding whether or not to allow us to test our finished product
with children later on in the project.

\subsection{Level design}
For this demonstration, we created a simple level which used counting
as its main objective. The player's goal was to collect twelve
dummy coins and deposit them in a hopper; once the player had collected
and deposited twelve coins in this way, the door at the far end of the level
would open, allowing the player to progress.

The level was designed to be as minimalistic as possible, so as to simply
showcase the potential for setting up problems in Minecraft which would
have to be solved mathematically. The level conveyed the idea that we could
build areas which players could not leave without first completing some
problem---in this case, counting.
