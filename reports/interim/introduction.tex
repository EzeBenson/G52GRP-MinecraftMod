\chapter{Introduction and background}
The EduCraft project is concerned with developing a series of
extensions\footnote{These are commonly called `mods'. Throughout the report, we will refer to our product as `the mod'} for the popular computer game
Minecraft, aimed at promoting collaborative learning in primary schools,
particularly with reference to numeracy education. In this first section, some
of the advantages of collaborative learning are explored in contrast to
conventional teaching methods, and the game of Minecraft is introduced in the
context of education.

\section{Minecraft}
\begin{quote}
``Minecraft is a game about breaking and placing blocks. At first, people
built structures to protect against nocturnal monsters, but as the game grew
players worked together to create wonderful, imaginative things.''
\cite{website:minecraft}
\end{quote}

Minecraft was initially developed in 2009, as a game where players could
explore a randomly-generated world and place blocks to build structures. From
the outset, Minecraft was not designed as a game that could be `won'---in
game industry terms, it was intended to be a `sandbox' game where players set
their own goals.

There have been a number of articles written exploring the potential for
Minecraft to be used in education \cite{brand13,short2012}. Habgood has
already established the usefulness of computer games in general in education
\cite{habgood2007}; the question is whether Minecraft can be used in the ways
that he suggests.
