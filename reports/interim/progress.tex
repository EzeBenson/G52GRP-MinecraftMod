\chapter{Progress and Problems}

\section{Introduction}
In this chapter we will describe a discussion of problems faced 
during the project so far, including both programming and people.

\section{Research}
Everyone did their due diligence and read Jacob Habgood’s thesis. 
Janos managed to find the original thesis in the library which 
included a cd of the game and source code as well. Unfortunately, 
we were not able to read the contents of the CD.

The next task was to get familiar with Minecraft. Only Ali and 
Ian had any previous experience playing it. We tasked ourselves 
with getting familiar with the game and Janos was very helpful 
with getting the minecraft environment set up for everyone. 

After gaining some understanding of  the features and limitations 
of minecraft. We researched what platform to use to develop our 
mod. We eventually decided to use MinecraftForge API for reasons 
discussed in the design/implementation section.

Eze brilliantly took it upon himself to research the KS2 Maths 
Curriculum from which he developed the first version of our 
requirements specification.

\section{Assigning Roles and Team Leader}
One of the very first tasks to accomplish was the assigning of 
roles and who would be team leader. Our group supervisor encouraged 
that the roles be switched around as the project continues. We 
decided to do this. Roles assigned for the first iteration are:

\begin{itemize}
\item Documentation - Tosin/Janos
\item Website - Tosin
\item Version Control - Tosin
\item Design Ideas - Ying
\item Coders - Eze/Ian
\item Testing - Ali
\end{itemize}

During the Formal Group Meetings, the role of chairman and 
secretary was assigned at the beginning of each meeting.

Choosing a Team Leader was the least trivial decision. No one 
person was keen on performing the role but Janos put himself 
up for it. The team has got along well so far so there hasn't 
been a crucial need for the team leader to use his authority.

\section{Generation of Ideas}
After concluding initial research and deciding roles, the team 
looked into implementing the actual game. The focus was to keep 
collaboration at the core of the game design. We took some 
inspiration from habgood thesis. 

We developed a core theme of how to integrate collaboration 
in the levels. Basically, there would be some sort of game 
mechanic whereas players would be able to earn mathematical 
operators and numbers. (E.g killing zombies in a non graphical 
way). The players would then solve a mathematical problem by 
combining the numbers and mathematical operators from each 
player. Upon completion of the problem, they would then all 
proceed to the next level.

\section{Web and Repo}
The version control system provided by the university is SVN. 
Most of the team was more familiar with Git and hence the team 
pushed to used Github. Permission was obtained to use Github 
as long as we kept some sort of backup with svn.

Tosin looked into converting Git commits and history into svn 
but it didn't seem very feasible. During this time, unfortunately 
no backup has been set up on the svn yet. Starting from the 
13th of december 2013, a weekly backup of the git repo will be 
made. This will include the git history allowing us to still 
revert commits and perform other commands incase the git repo 
on github ever fails.

The project site is hosted on code.cs.nott.ac.uk. The Project 
site was completed in due time and the team gained 3 out of a 
possible 5 marks. One of the main loss of marks was due to 
Tosin’s foresight as to status of the project description.

When Tosin was setting up the Project Site, he raised an issue 
that members of the team should look into contributing some 
more info to the project site, unfortunately this went unanswered. 
In hindsight, an easy way to improve the Project’s description 
would have been to add Eze’s initial system requirements.


\section{Prototyping a mod}
Before the christmas break, it was planned that we get a 
simple, one player mod of our project working. 
Progress on that has been quite good, with Ian contributing 
quite a bit of code.

/* someone expand on this */
